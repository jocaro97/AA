\documentclass[size=a4, parskip=half, titlepage=false, toc=flat, toc=bib, 12pt]{scrartcl}

\setuptoc{toc}{leveldown}

% Ajuste de las líneas y párrafos
\linespread{1.2}
\setlength{\parindent}{0pt}
\setlength{\parskip}{12pt}

% Español
\usepackage[spanish, es-tabla]{babel}

% Matemáticas
\usepackage{amsmath}
\usepackage{amsthm}

% Links
%\usepackage{hyperref}

% Fuentes
\usepackage{newpxtext,newpxmath}
\usepackage[scale=.9]{FiraMono}
\usepackage{FiraSans}
\usepackage[T1]{fontenc}

\defaultfontfeatures{Ligatures=TeX,Numbers=Lining}
\usepackage[activate={true,nocompatibility},final,tracking=true,factor=1100,stretch=10,shrink=10]{microtype}
\SetTracking{encoding={*}, shape=sc}{0}

\usepackage{graphicx}
\usepackage{float}

% Mejores tablas
\usepackage{booktabs}

\usepackage{adjustbox}

% COLORES

\usepackage{xcolor}

\definecolor{verde}{HTML}{007D51}
\definecolor{esmeralda}{HTML}{045D56}
\definecolor{salmon}{HTML}{FF6859}
\definecolor{amarillo}{HTML}{FFAC12}
\definecolor{morado}{HTML}{A932FF}
\definecolor{azul}{HTML}{0082FB}
\definecolor{error}{HTML}{b00020}

% ENTORNOS
\usepackage[skins, listings, theorems]{tcolorbox}

\newtcolorbox{recuerda}{
  enhanced,
%  sharp corners,
  frame hidden,
  colback=black!10,
	lefttitle=0pt,
  coltitle=black,
  fonttitle=\bfseries\sffamily\scshape,
  titlerule=0.8mm,
  titlerule style=black,
  title=\raisebox{-0.6ex}{\small RECUERDA}
}

\newtcolorbox{nota}{
  enhanced,
%  sharp corners,
  frame hidden,
  colback=black!10,
	lefttitle=0pt,
  coltitle=black,
  fonttitle=\bfseries\sffamily\scshape,
  titlerule=0.8mm,
  titlerule style=black,
  title=\raisebox{-0.6ex}{\small NOTA}
}

\newtcolorbox{error}{
  enhanced,
%  sharp corners,
  frame hidden,
  colback=error!10,
	lefttitle=0pt,
  coltitle=error,
  fonttitle=\bfseries\sffamily\scshape,
  titlerule=0.8mm,
  titlerule style=error,
  title=\raisebox{-0.6ex}{\small ERROR}
}

\newtcblisting{shell}{
  enhanced,
  colback=black!10,
  colupper=black,
  frame hidden,
  opacityback=0,
  coltitle=black,
  fonttitle=\bfseries\sffamily\scshape,
  %titlerule=0.8mm,
  %titlerule style=black,
  %title=Consola,
  listing only,
  listing options={
    style=tcblatex,
    language=sh,
    breaklines=true,
    postbreak=\mbox{\textcolor{black}{$\hookrightarrow$}\space},
    emph={jmml@UbuntuServer, jmml@CentOS},
    emphstyle={\bfseries},
  },
}

\newtcbtheorem[number within=section]{teor}{\small TEOREMA}{
  enhanced,
  sharp corners,
  frame hidden,
  colback=white,
  coltitle=black,
  fonttitle=\bfseries\sffamily,
  %separator sign=\raisebox{-0.65ex}{\Large\MI\symbol{58828}},
  description font=\itshape
}{teor}

\newtcbtheorem[number within=section]{prop}{\small PROPOSICIÓN}{
  enhanced,
  sharp corners,
  frame hidden,
  colback=white,
  coltitle=black,
  fonttitle=\bfseries\sffamily,
  %separator sign=\raisebox{-0.65ex}{\Large\MI\symbol{58828}},
  description font=\itshape
}{prop}

\newtcbtheorem[number within=section]{cor}{\small COROLARIO}{
  enhanced,
  sharp corners,
  frame hidden,
  colback=white,
  coltitle=black,
  fonttitle=\bfseries\sffamily,
  %separator sign=\raisebox{-0.65ex}{\Large\MI\symbol{58828}},
  description font=\itshape
}{cor}

\newtcbtheorem[number within=section]{defi}{\small DEFINICIÓN}{
  enhanced,
  sharp corners,
  frame hidden,
  colback=white,
  coltitle=black,
  fonttitle=\bfseries\sffamily,
  %separator sign=\raisebox{-0.65ex}{\Large\MI\symbol{58828}},
  description font=\itshape
}{defi}

\newtcbtheorem{ejer}{\small EJERCICIO}{
  enhanced,
  sharp corners,
  frame hidden,
  left=0mm,
  right=0mm,
  colback=white,
  coltitle=black,
  fonttitle=\bfseries\sffamily,
  %separator sign=\raisebox{-0.65ex}{\Large\MI\symbol{58828}},
  description font=\itshape,
  nameref/.style={},
}{ejer}

% CÓDIGO
\usepackage{listings}

% CABECERAS
\pagestyle{headings}
\setkomafont{pageheadfoot}{\normalfont\normalcolor\sffamily\small}
\setkomafont{pagenumber}{\normalfont\sffamily}

% ALGORITMOS
\usepackage[vlined,linesnumbered]{algorithm2e}
\usepackage{listings}
\usepackage{color}
\renewcommand{\lstlistingname}{Listado}

\definecolor{dkgreen}{rgb}{0,0.6,0}
\definecolor{gray}{rgb}{0.5,0.5,0.5}
\definecolor{mauve}{rgb}{0.58,0,0.82}

\lstset{frame=tb,
  language=Python,
  aboveskip=3mm,
  belowskip=3mm,
  showstringspaces=false,
  columns=flexible,
  basicstyle={\small\ttfamily},
  numbers=none,
  numberstyle=\tiny\color{gray},
  keywordstyle=\color{blue},
  commentstyle=\color{dkgreen},
  stringstyle=\color{mauve},
  breaklines=true,
  breakatwhitespace=true,
  tabsize=2
}

% Formato de los pies de figura
\setkomafont{captionlabel}{\scshape}
\SetAlCapFnt{\normalfont\scshape}
\SetAlgorithmName{Algoritmo}{Algoritmo}{Lista de algoritmos}

% BIBLIOGRAFÍA
%\usepackage[sorting=none]{biblatex}
%\addbibresource{bibliografia.bib}

\begin{document}

\renewcommand{\proofname}{\normalfont\sffamily\bfseries\small DEMOSTRACIÓN}

\title{Trabajo 3\\
Programación}
\subject{Aprendizaje automático}
\author{Johanna Capote Robayna\\
    5 del Doble Grado en Informática y Matemáticas\\
    Grupo A}
\date{}
\publishers{\vspace{2cm}\includegraphics[height=2.5cm]{UGR}\vspace{1cm}}
\maketitle

\newpage

\tableofcontents
\newpage

\section{Problema de clasificación}

\subsection{Problema a resolver}
El problema que se planea consiste en clasificar imágenes de los dígitos del sistema de numeración arábico. Para ello contamos con un dataset (\textit{Optical Recognition of Handwritten Digits Data Set}) en el que se encuentran imágenes ya procesadas de dígitos escritos a mano y nuestro objetivo es predecir dada una nueva imagen de que dígito se trata.

Para procesar las imágenes, en primer lugar cada imagen de 32x32 bits se divide en cuadrados de 4x4 no superpuestos. Cada bit indica si el píxel es blanco o negro. Dentro de cada cuadrado se cuenta el número de píxeles blancos y de píxeles negros, formando así una matriz de 8x8 donde cada elemento es un número entre 0 y 16. Por lo tanto por cada imagen tenemos 64 (8x8) características, donde cada característica es un número entre 0 y 16 que representa el número de píxeles negros o blancos que hay.

Esta base de datos se encuentra distribuida en dos ficheros en formato CSV \verb|optdigits.tra|, donde se encuentra el conjunto de datos de entrenamiento y \verb|optdigits.tes|, donde está el conjunto de \textit{test}. Cada dato consta de 64 características y una etiqueta entre el 0 y 9 que determina el dígito al que corresponden las imágenes. Por lo tanto identificamos los datos del problema como:
\begin{itemize}
\item $X$: vector entero de 64 características, cuyos valores se encuentran entre 0 y 16.
\item $Y$: valor entero entre 0 y 9 que índica de que dígito se trata.
\item $f$: función que hace corresponder cada vector de características $x$ con su etiqueta correspondiente $y$.
\end{itemize}

\newpage
\subsection{Selección de las clases de funciones a usar}


\subsection{Conjuntos \textit{training} y \textit{test}}


\subsection{Preprocesado de datos}

\subsection{Métrica de error}

\subsection{Técnica de ajuste}

\subsection{Regularización}

\subsection{Módelos}

\subsection{Estimación de hiperparámetros y elección del mejor modelo}

\subsection{Estimación por validación cruzada del error $E_{out}$}

\subsection{Empresa}

\newpage
\section{Problema de regresión}

\subsection{Problema a resolver}
\subsection{Selección de las clases de funciones a usar}
\subsection{Conjuntos \textit{training} y \textit{test}}
\subsection{Preprocesado de datos}
\subsection{Métrica de error}
\subsection{Técnica de ajuste}
\subsection{Regularización}
\subsection{Módelos}
\subsection{Estimación de hiperparámetros y elección del mejor modelo}
\subsection{Estimación por validación cruzada del error $E_{out}$}
\subsection{Empresa}

%printbibliography

\end{document}
